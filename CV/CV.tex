\documentclass[letterpaper]{article}

\usepackage{hyperref}
\usepackage{geometry}
\usepackage{multicol}
\usepackage{float}

\def\name{Miguel Cardoso Oliveira}

\def\footerlink{https://miguelcfoliveira.github.io/CV/CV.pdf}

\hypersetup{
  colorlinks = true,
  urlcolor = black,
  pdfauthor = {\name},
  pdfkeywords = {Corporate Finance, Financial Distress, Household Finance},
  pdftitle = {\name: Curriculum Vitae},
  pdfsubject = {Curriculum Vitae},
  pdfpagemode = UseNone
}

\geometry{
  body={6.5in, 8.5in},
  left=1.0in,
  top=1.25in
}


\pagestyle{myheadings}
\thispagestyle{empty}

\usepackage{sectsty}
\sectionfont{\rmfamily\mdseries\Large}
\subsectionfont{\rmfamily\mdseries\large}


\setlength\parindent{0em}

% Make lists without bullets
\newenvironment{itemize*}{
  \begin{list}{}{
    \setlength{\leftmargin}{1.5em}
  }
}{
  \end{list}
}

\begin{document}

% Place name at left
{\huge \name}

\vspace{0.3in}

\begin{minipage}{0.45\linewidth}
  \href{https://www.novasbe.unl.pt/en/}{Nova School of Business and Economics} \\
  Campus de Carcavelos, Rua da Holanda, n.º 1 \\ 
  2775-405 Carcavelos, Portugal 
\end{minipage}
\begin{minipage}{0.45\linewidth}
  \begin{tabular}{ll}
    Phone: & +351 969 830 905 \\
    Email: & \href{mailto:miguel.oliveira@novasbe.pt}{\tt miguel.oliveira@novasbe.pt} \\
    Homepage: & \href{https://miguelcfoliveira.github.io}{\tt https://miguelcfoliveira.github.io} 
  \end{tabular}
\end{minipage}

\vspace{0.3in}

\section*{References}
\begin{multicols}{2}
\textbf{Fernando Anjos} \\
Associate Professor of Finance \\
Nova School of Business \& Economics \\
Email: \href{mailto:fernando.anjos@novasbe.pt}{\tt fernando.anjos@novasbe.pt} 
\columnbreak \hfill

\textbf{Miguel A. Ferreira} \\
Vice-Dean of Faculty \& Research \\
Nova School of Business \& Economics \\
Email: \href{mailto:miguel.ferreira@novasbe.pt}{\tt miguel.ferreira@novasbe.pt} 
\end{multicols}

\begin{multicols}{2}
\textbf{Manuel Adelino} \\
Associate Professor of Finance \\
Fuqua School of Business, \\ 
Duke University \\
Email: \href{mailto:manuel.adelino@duke.edu}{\tt manuel.adelino@duke.edu} 
\columnbreak \hfill

\phantom{a}
\end{multicols}

\section*{Education}
\begin{itemize*}
\item \textbf{Ph.D., Economics \& Finance}, Nova School of Business and Economics \hfill 2019 - Present
\item \phantom{aaa} Visiting Ph.D. Student at Carnegie Mellon University (Spring 2023)
\item \textbf{Summer School in Structural Estimation} (Corporate Finance), Mitsui Center \hfill August 2021
\item \textbf{M.Sc., Finance}, Nova School of Business and Economics \hfill 2014 - 2016
\item \textbf{B.Sc., Economics}, Nova School of Business and Economics \hfill  2011 - 2014
\item \phantom{aaa} Erasmus Program at Luiss Guido Carli
\end{itemize*}

\section*{Research Interests}
\begin{itemize*}
\item Corporate Finance, Financial Distress \& Bankruptcy, Household Finance
\end{itemize*}

\section*{Job Market Paper}
\begin{itemize*}
\item \textbf{Homemade Unleverage: Do Households Care About Employers' Leverage?}
\item Abstract: \textit{Exploiting a rich dataset of matched households and employers, I provide novel evidence on the impact of the employer's capital structure on employees' consumption and saving decisions. Notwithstanding receiving lower wages, households working for highly leveraged employers exhibit lower marginal propensities to consume. This effect is driven by cutting in ``luxury'' goods and services, thus suggesting a novel channel through which financial distress costs spill over to other---potentially unrelated---firms: the employee-spending channel. To establish causality, I look at employees' responses to negative industry-wide shocks and find that only those employed by high-leverage firms cut consumption, though I find no differential effect on wages. I reconcile these facts with a Diamond-Mortensen-Pissarides matching model, in which heterogeneous risk-averse employees bargain with heterogeneous employers to determine wages. Consistent with the model, the consumption response is mainly driven by poorer households, for whom unemployment is more painful. Overall, evidence is suggestive that financial distress costs are being partially shifted to employees.}
\end{itemize*}

\section*{Working Papers}

\begin{itemize*}
\item \textbf{The Heterogeneous Effects of Household Debt Relief}

	(with Manuel Adelino and Miguel Ferreira)
\medskip
	
\item \textbf{Do Specialized Distress Investors Undermine Upstream Lending?}

	(with Fernando Anjos and Irem Demirci)

\end{itemize*}

\section*{Work in Progress}

\begin{itemize*}
\item \textbf{How costly is default around the world? Evidence from structural estimation}
\end{itemize*}

\section*{Seminars and Conference Presentations}
\vspace{-0.1in}
(* by co-author)
\begin{itemize*}
\item \textbf{``Do Specialized Distress Investors Undermine Firms Ex Ante?''}
\begin{itemize}
\item 2023: Vienna Festival of Finance Theory, Vienna*; Brownbag at Tepper School of Business - Carnegie Mellon University, Pittsburgh PA; SKEMA Conference on Corporate Restructuring, Nice*; Lubrafin, Braga*; Spanish Finance Forum, Malaga*
\item 2022: Cambridge-Nova Finance Workshop, Cambridge*
\end{itemize}
\item \textbf{``The Heterogeneous Effects of Household Debt Relief''}
\begin{itemize}
\item 2024: Stanford Institute for Theoretical Economics (SITE) Financial Regulation Session, Stanford*; NBER SI Capital Markets and the Economy, Cambridge, MA*; University of Tennessee ``Smokey'' Mountain Finance Conference, Townsend, TN*; CEPR European Workshop on Household Finance, London*
\item 2023: Spanish Finance Forum, Malaga; Annual Meeting of the Portuguese Economic Journal, Braga; 
\end{itemize}
\item \textbf{``Homemade Unleverage: Do Households Care About Employers' Leverage?''}
\begin{itemize}
\item 2025: AFA PhD Student Poster Session, San Francisco, CA
\item 2024: Nova SBE Finance PhD Final Countdown, Lisbon; Nova SBE Faculty Seminar
\end{itemize}
\end{itemize*}

\section*{Discussions}
\begin{itemize*}
\item Nova SBE Finance PhD Final Countdown, Lisbon 2024; Lubrafin, Braga July 2023; Spanish Finance Forum (AEFIN), Malaga July 2023; Cambridge Judge Business School - Nova SBE Workshop,  Cambridge 2022; Nova SBE Finance PhD Final Countdown, Lisbon 2022; Nova SBE Finance PhD Pitch Perfect, Lisbon 2022
\end{itemize*}

\section*{Academic Honors and Grants}

\begin{itemize*}
\item Ph.D. Scholarship, FCT (\textit{Fundação para a Ciência e Tecnologia}) \hfill 2019 - 2024
\item NOVA SBE Scholarship for merit (M.Sc. partial tuition waiver) \hfill 2014 - 2016
\end{itemize*}

\section*{Teaching Experience}
\begin{itemize*} 
\item \textbf{Nova School of Business and Economics}, Course Instructor
\begin{itemize*}
\item Corporate Finance (M.Sc. in Finance) \hfill Fall 2023 - Present
\begin{itemize*}
\item Teaching Evaluations: 5.6 (2023) 
\item Grading System: 6 (very good) to 1 (unsatisfactory)
\end{itemize*}
\end{itemize*}
\item \textbf{Nova School of Business and Economics}, Teaching Assistant
\begin{itemize*}
\item Applied Corporate Finance (M.Sc. in Finance) \hfill Spring 2023
\item Corporate Finance (M.Sc. in Finance) \hfill Fall 2022
\item Data Analytics for Finance (M.Sc. in Finance) \hfill Spring 2022
\item Small Business Management (M.Sc. in Management)  \hfill Fall 2021
\item Small Business Management (M.Sc. in Management)  \hfill Fall 2020
\item International Taxation (M.Sc. in Finance) \hfill Spring 2019
\item Principles of Management (B.Sc. in Economics/Management) \hfill 2016-2022
\end{itemize*}
\end{itemize*}

\section*{Service}
\begin{itemize*}
\item Organizer of the Nova SBE Final Countdown \hfill 2024
\item Developer of software for optimally allocate students and rooms to exams at Nova \hfill 2022-present
\item Nova Finance Knowledge Center (Member) \hfill 2021-present
\item Guest Lecturer at \textit{Programa Cascais Surf para a Empregabilidade} \hfill 2019-2023
\item Coordinator and Discussant of a case-solving student club (\textit{Nova Case Team}) \hfill 2017-2020
\end{itemize*}

\section*{Non-Academic Experience}

\begin{itemize*}
\item \textbf{Financial Consultant}, Reorganizations \& Bankruptcies w/ small Law firm \hfill 2016-2019
\item \textbf{Founder and Manager}, First Equity, Lda. (Commercial Real Estate) \hfill 2016-2019
\end{itemize*}

\section*{Skills}
\renewcommand{\arraystretch}{1.25}
\begin{table}[h]
 \vspace*{-\baselineskip}
\begin{tabular}{l l}
\textbf{Computer Skills} & Python, Stata, Matlab, \LaTeX, Microsoft Office \\
\textbf{Languages} & Portuguese (native), English (fluent), Italian (basic)
\end{tabular}
\end{table}

% Footer
\begin{center}
  \begin{footnotesize}
    Last updated: \today \\
    \href{\footerlink}{\texttt{Link to latest version}}
  \end{footnotesize}
\end{center}

\end{document}
